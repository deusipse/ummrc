\section{Generalisation}

We were able to produce a generalisation of Problem 2:
\begin{itemize}
  \item Suppose that if $m$ X-particles are next to each other, they collapse into $n$ X-particles.
  \item Assume that $k$ particles have been emitted.
  \item The probability of firing an X-particle is $p$.
  \item Find the average number of particles.
  \item It is assumed that $m > n \ge 1$, otherwise the number of particles would simply blow up infinitely.
\end{itemize}

\begin{theorem}
  The average number of particles after $k$ shots, when $m$ X-particles collapse into $n$, is $k$ when $k < m$, and is 
  \begin{equation}
    m-1+\sum _{b=m-1}^{k-1} \left(1-p \left(\sum _{a=0}^{\floor*{\frac{b-m}{m-n}} } (1-p) p^{a (m-n)+m-1}+\varepsilon\right) (m-n)\right) \label{formula}
  \end{equation}
  when $k \ge m$ and $\varepsilon = \begin{cases} p^{b}, &b-m+1 = 0 \bmod m-n \\ 0, &b-m+1 \neq 0 \bmod m-n. \end{cases}$
\end{theorem}
\begin{proof}
  Let the average number of particles after $k$ shots be denoted by $T_k$. It is obvious that $T_k = k$ when $k < m$, as it is impossible for any particles to have collapsed. Thus we must proceed to find a formula for $T_k$ when $k \ge m$. We can achieve this by establishing a recurrence for $T_k$ in order to find the difference between each term, and from there we may simply sum the difference to obtain the required formula. 

  We know that $T_k$ is recursive since there are two events that can occur after each shot:
  \begin{itemize}
    \item Number of particles increases by 1
    \item Number of particles increases by $n-m+1$
  \end{itemize}
  This is because if an X-particle were added to $m-1$ consecutive X-particles, they would collapse into $n$ X-particles. Hence, the number of particles `increases' by $n-m+1$, although this number is always negative or 0 (as $m > n$). Thus it remains to determine the probability that there are $m-1$ consecutive X-particles at the end of the tube. Letting this probability be $\vartheta$, the probability of the particles collapsing on the next shot is simply $p\vartheta$. As such, the expected number of particles increases by 1 with a probability of  $1-p\vartheta$ and increases by  $n-m+1$ with a probability of $p\vartheta$. Hence, the recurrence is:
  \begin{align}
    T_{k+1} &= T_{k} + 1 - p\vartheta + p\vartheta(n-m+1) \nonumber \\
            &= T_{k} + 1 - p\vartheta(m-n). \label{eq:1}
  \end{align}
  The remaining step is to calculate $\vartheta$.
  \begin{claim}
    \begin{equation*}
      \vartheta = \sum_{a=0}^{\floor*{\frac{k-m}{m-n}}} (1-p)p^{a(m-n)+m-1} + \varepsilon,
    \end{equation*}
    where $\varepsilon = \begin{cases} p^{k}, &k-m+1 = 0 \bmod m-n \\ 0, &k-m+1 \neq 0 \bmod m-n. \end{cases}$
  \end{claim}
  \begin{proof}
    Recall that $\vartheta$ is the probability that there are $m-1$ consecutive X-particles at the end of the tube. In order to `count' the number of consecutive X-particles, there must be some end to the string of X-particles. This `end' can occur in two ways: either there is a Y-particle before the last X-particle; or there are no Y-particles at all, and the tube compromises completely of X-particles.
    \begin{figure}[H]
      \vspace{-1.5em}
      \begin{align*}
        &\underbrace{\text{X}\dots\text{X}}_{m-1}\text{Y} \tag{A}\\
        &\underbrace{\text{X}\dots\text{X}}_{m-1} \tag{B}
      \end{align*}
      \vspace{-2em}
    \end{figure}
    Let us first consider the case with a Y-particle at the end (configuration A). The probability of such a configuration occurring is simply $(1-p)p^{m-1}$. However, this probability does not account for possible previous collapses. For example, the following sequence of particles may have been emitted which would lead to the same configuration:
    \begin{figure}[H]
      \vspace{-1.5em}
      \begin{equation*}
        \underbrace{\text{X}\dots\text{X}}_{m + (m-n) - 1}\text{Y} \quad \longrightarrow \quad \underbrace{\text{X}\dots\text{X}}_{n+(m-n)-1}\text{Y}\quad = \quad \underbrace{\text{X}\dots\text{X}}_{m-1}\text{Y} 
      \end{equation*}
      \vspace{-2em}
    \end{figure}
    It is clear that adding any multiple of $m-n$ X-particles to $m-1$ consecutive X-particles results in the exact same configuration. However, since only $k$ particles have been fired, $a(m-n) + m-1 + 1 \le k$ for some positive integer $a$. We solve this inequality for $a$, getting
    \begin{align*}
      a(m-n) + m &\le k \\
      a(m-n) &\le k - m \\
      a &\le \frac{k-m}{m-n}.
    \end{align*}
    Thus the maximum value of $a$ is $\floor*{\dfrac{k-m}{m-n}},$ so the probability of configuration A is \[
      \sum_{a=0}^{\floor*{\frac{k-m}{m-n}}} (1-p)p^{a(m-n)+m-1}
    .\]
    We now consider configuration B, which consists of X-particles only. Again, the configuration can be achieved with $a(m-n) + m-1$ consecutive X-particles, for some positive integer $a$. However, this time there cannot be any other particles other than these X-particles, meaning that  $k = a(m-n) + m-1$. It now becomes obvious that $k-m+1$ must be an integer multiple of $m-n$, at which point there is a single sequence of particles which results in $m-1$ consecutive X-particles. When $k-m+1$ is not an integer multiple of $m-n$, there is no possible way for configuration B to exist. Thus the probability of configuration B occurring can be expressed by $\varepsilon$, where \[
      \varepsilon = 
      \begin{cases}
        p^{k}, &k-m+1 = 0 \bmod m-n \\
        0, &k-m+1 \neq 0 \bmod m-n.
      \end{cases}
    \] We then add the probabilities of configurations A and B to get the overall probability of the tube ending in $m-1$ consecutive X-particles, which is  \[
    \vartheta = \sum_{a=0}^{\floor*{\frac{k-m}{m-n}}} (1-p)p^{a(m-n)+m-1} + \varepsilon
    ,\] as required.
  \end{proof}
  Since we have previously determined a recursive formula for $T_k$ in terms of $\vartheta$ and $\varepsilon$ in Equation~\ref{eq:1}, we can simply sum the difference between each term to obtain a closed expression for the value of any term. We have \[
    T_{k+1} = T_k + 1 - p\vartheta(m-n)
  ,\] meaning the difference between terms is $1-p\vartheta(m-n)$. We then add $m-1$ to the sum of this difference from $m-1$ to $k$ (since the formula is only required for $k \ge m$). This yields \[
    T_k = m-1+\sum _{b=m-1}^{k-1} \left(1-p \left(\sum _{a=0}^{\floor*{\frac{b-m}{m-n}} } (1-p) p^{a (m-n)+m-1}+\varepsilon\right) (m-n)\right)
  . \tag*{\qedhere}\] 
\end{proof}
It should be noted that $\vartheta$ is the sum of a geometric sequence, although for brevity, the sum is left unexpanded. The expanded sum is  \[
  \vartheta = \frac{(p-1) p^{m+n-1} \left(p^{(m-n) \left(\floor*{\frac{k-m}{m-n}} +1\right)}-1\right)}{p^n-p^m}
.\] In addition, there also exists a closed form of $\varepsilon$, though it is somewhat distasteful. If we define a summation as 0 when its upper bound is less than its lower bound, then we obtain \[
\varepsilon = \sum_{a=0}^{(m-k-1) + \floor*{\frac{k-m+1}{m-n}}\cdot (m-n)} p^{k}
.\] The upper bound is simply the negative remainder of $k-m+1$ when divided by  $m-n$, which is 0 if and only if $k-m+1$ is an integer multiple of $m-n$, and negative everywhere else. The lower bound, $a$ is 0 simply to make the sum 0 whenever the upper bound is negative, and is not used elsewhere.

\subsection{An interesting special case}
The general formula for the average length in Equation~\ref{formula} is rather unwieldy. However, observe that many of the terms depend on $m-n$. If we set  $m-n = 1 \implies n = m - 1$, then the formula can be reduced dramatically:
\begin{align}
  T_k &= m-1+\sum _{b=m-1}^{k-1} \left(1-p \left(\sum _{a=0}^{\floor*{\frac{b-m}{1}} } (1-p) p^{a (1)+m-1}+\varepsilon\right) (m-n)\right) \nonumber \\
      &= m-1+\sum _{b=m-1}^{k-1} \left(1-p \left(\sum _{a=0}^{b-m} (1-p) p^{a+m-1}+ p^{b} \right) \right) \label{eq:3}
\end{align}
Notice that even $\varepsilon = p^{b}$ becomes constant, as $z = 0 \bmod 1$ for any integer  $z$. Let us now evaluate $\vartheta$, which is the sum of a geometric sequence:
\begin{align*}
  \vartheta &= \sum_{a=0}^{k-m} (1-p)p^{a+m-1} + p^k \\
            &= \sum_{a=0}^{k-m} (1-p)\bigl(p^{m-1}\bigr)p^{a} + p^k \\
            &= (1-p)\bigl(p^{m-1}\bigr)\left(\frac{1-p^{k-m+1}}{1-p}\right) + p^k \\
            &= p^{m-1}\bigl(1-p^{k-m+1}\bigr) + p^k \\
            &= p^{m-1} - p^{k} + p^{k} \\
            &= p^{m-1}
\end{align*}
This is dependent only on $m$, meaning $\vartheta$ is constant. Hence the difference between the terms of $T_k$ is constant when $n = m-1$, and so $T_k$ would be linear. Let us now compute this difference more precisely. Using Equation~\ref{eq:3}, we see that the difference is \[
  1-p\vartheta = 1-p^{m}
.\] The formula for $T_k$ is then simply 
\begin{align}
  T_k &= m-1 + \sum_{b=m-1}^{k-1} \left(1-p^m\right) \nonumber \\
      &= m-1 + (k-m+1)\left(1 - p^m\right). \label{niceform}
\end{align}
Using Equation~\ref{niceform}, we can easily derive the formula for $T_k$ when $m = 2$, $n = 1$ and $p = \frac{1}{2}$, which is equivalent to Problem~\hyperlink{p2}{2}: 
\begin{align*}
  T_k &= 2 - 1 + (k-2+1)\left(1 - \frac{1}{4}\right) \\
      &= 1 + \frac{3}{4}(k-1) \\
      &= \frac{3}{4}k + \frac{1}{4}.
\end{align*}
\subsection{Limiting difference between terms}
We now consider the difference between the terms, as the number of shots, $k$, tends towards infinity. Recall that the difference is 
\begin{equation}\label{diff}
  T_{k+1} - T_{k} = 1 - p\vartheta(m-n),
  \end{equation}
  where \[
\vartheta = \sum_{a=0}^{\floor*{\frac{k-m}{m-n}}} (1-p)p^{a(m-n)+m-1} + \varepsilon
\] and \[
  \varepsilon = 
  \begin{cases}
    p^{k}, &k-m+1 = 0 \bmod m-n \\
    0, &k-m+1 \neq 0 \bmod m-n.
  \end{cases}
\] As $k$ increases, $\varepsilon$ tends towards 0 since its only non-zero value is $p^{k}$. In addition, $\floor*{\frac{k-m}{m-n}}$ tends towards infinity. We hence have the limit as $k$ approaches infinity of \[
  \vartheta = \sum_{a=0}^{\infty} (1-p)p^{a(m-n)+m-1}
.\] Evaluating this geometric series yields
\begin{align*}
  \vartheta &= \sum_{a=0}^{\infty} (1-p)\bigl(p^{m-1}\bigr)\bigl(p^{m-n}\bigr)^{a} \\
            &= \frac{(1-p)p^{m-1}}{1-p^{m-n}}.
\end{align*}
Plugging this into Equation~\ref{diff} gives us
\begin{align}
  \lim_{k \to \infty} T_{k+1} - T_{k} &= 1 - p\left( \frac{(1-p)p^{m-1}}{1-p^{m-n}} \right) (m-n) \nonumber \\
                                      &= 1 - \left( \frac{p^{m} - p^{m+1}}{1-p^{m-n}} \right) (m-n). \label{eq:4}
\end{align}
Therefore the difference converges to a constant as the number of shots increases, implying that $T_k$ becomes more and more linear as the number of terms increases.

For example, let us calculate this limiting difference for $m = 4, n = 1, p = \frac{1}{2}$. Plugging in these values, we obtain $\frac{25}{28}$. By plotting the difference between terms, we see that the difference does indeed converge to this value (Figure~\ref{conv}).

We can also plot the average number of particles as the number of shots increases, whilst varying $p$. Obviously, when $p = 0$, the line is simply linear. However, things get more interesting as $p$ gets very close to 1. A plot is shown in Figure~\ref{coolgraph}, in which the average number of particles can be seen oscillating when $p = 1$. This oscillation is caused by the constant emission of X-particles, which causes it to collapse every $m-n$ shots. In addition, notice that $0 \le p \le 0.5$ produce very similar values.
\begin{figure}[H]
  \centering
  \begin{tikzpicture}
    \begin{axis}[width = 0.8*\textwidth, title = {$T_{k+1} - T_{k}$ vs $k$, $m = 4, n = 1, p = \frac{1}{2}$}, legend pos = south east, xlabel = $k$]
      \addplot[red] table[x=k, y=diff] {data.dat};
      \addplot[domain = 3:20, dashed] {25/28};
      \legend{$T_{k+1} - T_{k}$, $\frac{25}{28}$};
    \end{axis}
  \end{tikzpicture}
  \caption{}
  \label{conv}
\end{figure}
\begin{figure}[H]
  \centering
  \begin{tikzpicture}
    \begin{axis}[width = 0.9*\textwidth, title = {Average number of particles vs $k$, colour corresponds to $p$, $m = 5, n = 3$}, legend pos = south east, xlabel = $k$]
      \addplot[red!0!green] table[x=k, y=0] {data2.dat};
      \addplot[red!10!green] table[x=k, y=0.1] {data2.dat};
      \addplot[red!20!green] table[x=k, y=0.2] {data2.dat};
      \addplot[red!30!green] table[x=k, y=0.3] {data2.dat};
      \addplot[red!40!green] table[x=k, y=0.4] {data2.dat};
      \addplot[red!50!green] table[x=k, y=0.5] {data2.dat};
      \addplot[red!60!green] table[x=k, y=0.6] {data2.dat};
      \addplot[red!70!green] table[x=k, y=0.7] {data2.dat};
      \addplot[red!80!green] table[x=k, y=0.8] {data2.dat};
      \addplot[red!90!green] table[x=k, y=0.9] {data2.dat};
      \addplot[red!100!green] table[x=k, y=1] {data2.dat};
    \end{axis}
  \end{tikzpicture}
  \caption{Pure green corresponds to $p = 0$, pure red corresponds to $p = 1$}
  \label{coolgraph}
\end{figure}
