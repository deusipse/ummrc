A particle generator is emitting two types of particles (called X and Y) into a long tube. The particles will line up in order after entering the tube. Initially, the tube is empty. At each shot, either an X- or Y-particle is randomly emitted into the tube with equal probability. Different shots are assumed to be independent from each other. Suppose that $n$ shots have been emitted.

\section{Problem 1}

\begin{itemize}
  \item What is the probability that no two X-particles are next to each other?
\end{itemize}

\begin{theorem}
  The probability that no two X-particles are next to each other after $n$ shots is given by \[
    \frac{F_{n+2}}{2^n},
\] where $F_n$ is the $n$th Fibonacci number.
\end{theorem}
\begin{proof}
The probability we require can be calculated by dividing the total number of ways to arrange the contents of the tube such that there are no consecutive X-particles, by the total number of arrangements of the particles. That is to say:
\begin{equation*}
  \prob(\text{No consecutive X-particles}) = \frac{\# \text{Arrangements w/o consecutive X-particles}}{\# \text{Total arrangements}}
\end{equation*}
\begin{claim}\label{combiformula}
  The number of arrangements with no consecutive X-particles is 
  \begin{equation}
    \sum_{k=0}^n \binom{n-k+1}{k}.
  \end{equation}
\end{claim}
\begin{proof}
  Consider a tube with $n$ particles in it. Let the number of X-particles be equal to $k$, so that the number of Y-particles is $n-k$. Consider the tube without the X-particles, consisting solely of Y-particles in a line:
  \begin{equation*}
  \underbrace{\text{YY}\dots\text{YY}}_{n-k}
  \end{equation*}
  Now consider the `gaps' between these Y-particles, indicated by a bar:
  \begin{equation*}
    \text{|Y|Y|\dots |Y|Y|}
  \end{equation*}
  Notice that there are exactly $n-k+1$ `gaps'. Clearly, if we were to only place (single) X-particles in the gaps, then there would never be any consecutive X-particles. This can be done in a total of \[
    \binom{n-k+1}{k}
    \] ways. However, we must consider this for any number of X-particles $k$, so we arrive at the sum \[
    \#\text{Arrangements with no consecutive X-particles} = \sum_{k=0}^n \binom{n-k+1}{k}. \qedhere
  \]
\end{proof}
\begin{claim}\label{combithing}
  \[
    \sum_{k=0}^n \binom{n-k+1}{k} = F_{n+2},
  \] where $F_n$ is the $n$th Fibonacci number.
\end{claim}
\begin{proof}
  Figure~\ref{fig:1} showcases a way to obtain the identity from Pascal's triangle. We will present an algebraic proof below.
  \begin{figure}[H]
    \centering
    \begin{asy}
      unitsize(0.5cm);

      int fib(int n) {
        if (n == 0) 
        return 0;
        else if (n == 1) 
        return 1;
        int fibs[] = {0, 1};
        while (fibs.length <= n) {
          fibs.push(fibs[fibs.length - 1] + fibs[fibs.length - 2]);
        }
        return fibs[fibs.length - 1];
      }

      int rows = 6;

      for (int i = 0; i <= rows; ++i) {
        pair A = (-i, -i*sqrt(3));
        pair B = (3, sqrt(3));
        path idk = A -- (A + 0.5*i*B) + 0.5*B;
        draw(L = Label(string(fib(i+1)), position = EndPoint), idk, blue+opacity(0.4));
      }

      for (int i = 0; i < floor(rows/2)+1; ++i) {
        pair A = (-rows, -rows*sqrt(3));
        pair B = (A + i*(3, sqrt(3)));
        if (i != 0)
        draw(B + 0.25*(-1, sqrt(3)) -- B + 0.75*(-1, sqrt(3)), red);
        if (i != floor(rows/2))
        draw(B + 0.25*(1, sqrt(3)) -- B + 0.75*(1, sqrt(3)), red);
      }

      for (int i = 0; i <= rows; ++i) {
        for (int j = 0; j <= i; ++j) {
          label(string(choose(i, j)), (j*2 - i, -i*sqrt(3)));
        }
      }
    \end{asy}
    \caption{Each number in the row is the sum of the numbers in the previous two rows.}
    \label{fig:1}
  \end{figure}

  Recall that the Fibonacci numbers are defined as follows:
  \begin{align*}
    &F_0 = 0, \\
    &F_1 = 1, \\
    &F_n = F_{n-1} + F_{n-2}. \tag{$n > 1$}
  \end{align*}
  We will define a function $f(n) = \sum_{k=0}^n \binom{n-k+1}{n}$. It is sufficient to prove that $f(1) = F_{3} = 2$, $f(2) = F_{4} = 3$, and that $f(n) = f(n-1) + f(n-2)$, which would then imply the result by definition of the Fibonacci numbers.

  It is obvious that $f(1) = \binom{2}{0} + \binom{1}{1} = 2$, which is equal to $F_{3}$. Next, $f(2) = \binom{3}{0} + \binom{2}{1} + \binom{1}{2} = 3$. Notice that we define $\binom{n}{k} = 0$ when $n < k$, as it is impossible to choose $k$ things from a set with less than $k$ elements.

  We proceed to prove that $f(n) = f(n-1) + f(n-2)$, where $n > 2$. Using the fact that $\binom{n}{0} = 1$, we rewrite $f(n)$ using Pascal's identity and linearity as \[
    f(n) = 1 + \sum_{k=1}^n \binom{n-k+1}{k} = 1 + \sum_{k=1}^n \binom{n-k}{k} + \sum_{k=1}^n \binom{n-k}{k-1}.
  \] Next, we simplify, getting
  \begin{align*}
    f(n) &= \sum_{k=0}^n \binom{n - k}{k} + \sum_{k=1}^n \binom{n-k}{k-1} \\
         &= \sum_{k=0}^{n-1} \binom{n-k}{k} + \binom{n-n}{k} + \sum_{k=1}^n \binom{n-k}{k-1} \\
         &= \sum_{k=0}^{n-1} \binom{n-k}{k} + 0 + \sum_{k=0}^{n-1} \binom{n-k-1}{k} \\
         &= \sum_{k=0}^{n-1} \binom{n-k}{k} + \sum_{k=0}^{n-2} \binom{n-k-1}{k} + \binom{n-(n-1)-1}{n-1} \\
         &= f(n-1) + f(n-2) + 0.
  \end{align*}
  Hence $f(n)$ must be equivalent to $F_{n+2}$.
\end{proof}
In the spirit of Arthur Benjamin and Jennifer Quinn\cite{downfall}, we also provide a purely combinatorial proof for Claim~\ref{combithing}.
\begin{proof}
  Let the number of arrangements without consecutive X-particles after $n$ shots be $R_n$. Now consider the two choices of the first particle: either Y or X. If the first particle was a Y-particle, then the number of arrangements without consecutive X-particles is simply $R_{n-1}$. However, if the first particle was an X-particle, then the next particle must be a Y-particle, and therefore the number of arrangements is $R_{n-2}$. Hence $R_{n} = R_{n-1} + R_{n-2}$, and since $R_1 = 2$ and  $R_2 = 3$, we have $R_{n} = F_{n+2}$. We then equate this with the formula obtained in Claim~\ref{combiformula} to conclude the proof.
\end{proof}
\begin{claim}
  The number of total arrangements of a tube with $n$ particles is \[2^n.\]
\end{claim}
\begin{proof}
  Each particle in the tube can be either an X-particle or a Y-particle, meaning there are 2 choices for each of the $n$ particles. Hence, there are a total of $2^n$ arrangements.
\end{proof}
Hence by dividing the number of arrangements where there are no consecutive X-particles by the total number of arrangements, we arrive at the formula \[\frac{F_{n+2}}{2^n}\] which gives the desired probability.
\end{proof}

