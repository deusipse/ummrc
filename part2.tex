\section{Problem 2}
\hypertarget{p2}{}
Now, if two X-particles are next to each other, they will immediately collide into one single X-particle.
\begin{itemize}
  \item Compute the average number of particles for $n = 2, 3, 4$.
  \item Can you find the pattern and establish an explicit formula for general $n$?
\end{itemize}

We found it most straightforward to proceed directly to finding a general formula. However, it should be noted that it is relatively simple to compute the averages for small $n$ by considering every possible tube after $n$ shots.

\begin{theorem}\label{thm:2}
  The average number of particles after $n$ shots is \[
  \frac{3}{4}n + \frac{1}{4}
  .\]
\end{theorem}
\begin{proof}
  Let the average number of particles after $n$ shots be $T_n$. Obviously, $T_1 = 1$. Next, consider the chance that after firing a shot, the number of particles \emph{doesn't} increase. This occurs only in the event that the last particle in the tube is an X-particle, and when the particle emitted is also an X-particle. Since both events have a probability $1 / 2$, the probability that both occur is simply $1 / 4$. 
  \begin{figure}[H]
    \vspace{-1.5em}
    \begin{align*}
    &\text{X} \to \text{X}: \text{X} \\
    &\text{Y} \to \text{X}: \text{YX} \\
    &\text{X} \to \text{Y}: \text{XY} \\
    &\text{Y} \to \text{Y}: \text{YY}
    \end{align*}
    \vspace{-3em}
  \end{figure}
  Hence, the probability that the number of particles \emph{does} increase after firing a shot is $1 -1 / 4 = 3 / 4$. Thus the expected number of particles increases by $3 / 4$ after each shot, giving us the recursion \[
    T_{n+1} = T_{n} + \frac{3}{4}
  .\] Since $T_1 = 1$, we arrive at the formula \[
    T_{n} = \frac{3}{4}n + \frac{1}{4} \tag*{\qedhere}
  .\] 
\end{proof}

Using this, we can easily compute the average number of particles when $n = 2, 3, 4$:
\begin{table}[H]
  \centering
  \begin{tabular}{cc}
    \toprule
    $n$ & $T_n$ \\
    \midrule
    1 & 1 \\
    2 & 1.75 \\
    3 & 2.5 \\
    4 & 3.25 \\
    \bottomrule
  \end{tabular}
\end{table}

