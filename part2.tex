\section{Problem 2}

Compute the average number of particles for $n = 2, 3, 4$.

Can you find the pattern and establish an explicit formula for general $n$?

\begin{theorem}\label{thm:2}
  The average number of particles after $n$ shots is \[
  0.75n + 0.25
  .\]
\end{theorem}
\begin{proof}
  Let the average number of particles after $n$ shots be $T_n$. Obviously, $T_1 = 1$. Next, consider the chance that after firing a shot, the number of particles \emph{doesn't} decrease. This occurs only in the event that the last particle in the tube is an X-particle, and when the particle emitted is also an X-particle. Since both events have a probability $0.5$, the probability that both occur is simply $0.25$. Hence, the probability that the number of particles \emph{does} increase after firing a shot is $1 - 0.25 = 0.75$. Thus the expected number of particles increases by $0.75$ after each shot, giving us the recursion \[
    T_{n+1} = T_{n} + 0.75
  .\] Since $T_1 = 1$, we arrive at the formula \[
    T_{n} = 0.75n + 0.25 \tag*{\qedhere}
  .\] 
\end{proof}
